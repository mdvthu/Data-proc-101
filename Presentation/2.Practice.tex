%% This presentation was based on Tim Osgood's
%% excellent freely available template:
%% timhosgood@gmail.com %%
%% Thank you

\documentclass{beamer}
\usepackage[utf8]{inputenc}
\usepackage{charter}
\usepackage{tikz}
\usepackage{graphicx}
\usepackage{amsmath}
\usepackage{amssymb}
\usepackage{epigraph}
\usepackage{trace}

\usepackage{listings}

%% Title slide formatting %%

\pgfdeclareimage[width=\paperwidth]{titlebackground}{images/title-slide-background.png}
\setbeamerfont{subtitle}{size=\tiny}
\setbeamertemplate{title page}{
	\begin{picture}(0,0)
		\put(-28.5,-163){%
			\pgfuseimage{titlebackground}
		}
		\put(0,-75){%
			\begin{minipage}[b][4.5cm][t]{0.5\textwidth}
				\color{black}
				\usebeamerfont{title}
				{\inserttitle\\[0.9cm]}
				\usebeamerfont{subtitle}
				{\insertauthor\par}
				{\insertinstitute\\[0.3cm]}
				{\insertdate}
			\end{minipage}
		}
	\end{picture}
}



%% General slide formatting %%

\definecolor{nhsdarkblue}{RGB}{0,48,135}

\pgfdeclareimage[width=1.4cm]{phntlogo}{images/phnt-logo.png}

\setbeamertemplate{headline}
{%
	\begin{picture}(0,0)
		\put(314,-30){%
			\pgfuseimage{phntlogo}
		}
		\put(20,-55){%
			\rule{320pt}{0.4pt}
		}
	\end{picture}
}

\setbeamertemplate{frametitle}
{%
	\begin{picture}(0,0)
		\put(-8,-10){%
			\normalsize\color{nhsdarkblue}\insertframetitle
		}
		\put(-7,-20){%
			\tiny\color{nhsdarkblue}\insertframesubtitle
		}
	\end{picture}
}

\setbeamertemplate{footline}
{%
	\begin{picture}(0,0)
		\put(20,30){%
			\rule{320pt}{0.4pt}
		}
		\put(100,14){%
			\color{nhsdarkblue}\insertshortdate
		}
		\put(160,14){%
			\color{nhsdarkblue}\insertshorttitle
		}
		\put(337,14){%
			\color{nhsdarkblue}\insertpagenumber
		}
	\end{picture}%
}



%% Information (author, title, etc.) %%

\title[Programming workshop]{Programming workshop: a very brief introduction}

\author{
	\sc{Mark Thurston}\\
    \textit{Radiology Registrar}
} 
\institute{
	\textit{Peninsula Radiology Academy}\\
    \textit{Plymouth}
}
\date[April 2018]{An introduction to automation with Python} % short date for footer



%% Content of slides %%

\begin{document}

    \begin{frame}[plain]
	    \titlepage
    \end{frame}


    \begin{frame}
	    \frametitle{Session aims}
	    \framesubtitle{By the end of this session, you will be able to:}

	    \begin{itemize}
		    \item Part 2: Practice
			    \begin{itemize}
				    \item set up a Python environment
				    \item run some pre-written example programs
				    \item Write your own simple example programs
			    \end{itemize}
	    \end{itemize}
	    \begin{itemize}
		    \item questions welcome throughout
	    \end{itemize}
    \end{frame}
    %


    \begin{frame}
	    \frametitle{Installing Python}
	    \framesubtitle{Getting up and running}

	    \begin{itemize}
		    \item Python is completely free to run on your computer without restrictions
		    \item no disadvantages to installation: no privacy concerns I'm aware of
		    \item download link:
			    \begin{itemize}
				    \item \url{https://www.python.org/downloads/}
			    \end{itemize}
		    \item 64-bit version should run on all reasonably modern computers
			    \begin{itemize}
		    \item try the 32-bit version if unsuccessful
			    \end{itemize}
	    \end{itemize}
    \end{frame}
    %
    
    
    \begin{frame}
	    \frametitle{Starting the Python interpreter}
	    \framesubtitle{Getting started with the Python interpreter}

	    \begin{itemize}
		    \item Applications $\Rightarrow$ Utilities $\Rightarrow$ Terminal
		    \item Finder window, Go: Utilities $\Rightarrow$ Terminal
		    \item type \textit{``python3''} at the \$ prompt
	    \end{itemize}
    \end{frame}
    %

\defverbatim[colored]\lstA{
	\begin{lstlisting}[language=Python,basicstyle=\ttfamily,keywordstyle=\color{red}]
	# full definition
	print( *objects, sep=' ', end='\n',
	file=sys.stdout, flush=False)

	# most useful part
	print( *objects )
	\end{lstlisting}

}



    \begin{frame}
	    \frametitle{Your first program}
	    \framesubtitle{Hello world}

	    \begin{itemize}
		    \item Which function do you need?
	    \end{itemize}
	    \lstA
    \end{frame}
    %









\end{document}
