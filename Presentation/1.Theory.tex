%% This presentation was based on Tim Osgood's
%% excellent freely available template:
%% timhosgood@gmail.com %%
%% Thank you

\documentclass{beamer}
\usepackage[utf8]{inputenc}
\usepackage{charter}
\usepackage{tikz}
\usepackage{graphicx}
\usepackage{amsmath}
\usepackage{amssymb}
\usepackage{epigraph}
\usepackage{trace}

\usepackage{listings}

%% Title slide formatting %%

\pgfdeclareimage[width=\paperwidth]{titlebackground}{images/title-slide-background.png}
\setbeamerfont{subtitle}{size=\tiny}
\setbeamertemplate{title page}{
	\begin{picture}(0,0)
		\put(-28.5,-163){%
			\pgfuseimage{titlebackground}
		}
		\put(0,-75){%
			\begin{minipage}[b][4.5cm][t]{0.5\textwidth}
				\color{black}
				\usebeamerfont{title}
				{\inserttitle\\[0.9cm]}
				\usebeamerfont{subtitle}
				{\insertauthor\par}
				{\insertinstitute\\[0.3cm]}
				{\insertdate}
			\end{minipage}
		}
	\end{picture}
}



%% General slide formatting %%

\definecolor{nhsdarkblue}{RGB}{0,48,135}

\pgfdeclareimage[width=1.4cm]{phntlogo}{images/phnt-logo.png}

\setbeamertemplate{headline}
{%
	\begin{picture}(0,0)
		\put(314,-30){%
			\pgfuseimage{phntlogo}
		}
		\put(20,-55){%
			\rule{320pt}{0.4pt}
		}
	\end{picture}
}

\setbeamertemplate{frametitle}
{%
	\begin{picture}(0,0)
		\put(-8,-10){%
			\normalsize\color{nhsdarkblue}\insertframetitle
		}
		\put(-7,-20){%
			\tiny\color{nhsdarkblue}\insertframesubtitle
		}
	\end{picture}
}

\setbeamertemplate{footline}
{%
	\begin{picture}(0,0)
		\put(20,30){%
			\rule{320pt}{0.4pt}
		}
		\put(100,14){%
			\color{nhsdarkblue}\insertshortdate
		}
		\put(160,14){%
			\color{nhsdarkblue}\insertshorttitle
		}
		\put(337,14){%
			\color{nhsdarkblue}\insertpagenumber
		}
	\end{picture}%
}



%% Information (author, title, etc.) %%

\title[Programming workshop]{Programming workshop: a very brief introduction}

\author{
	\sc{Mark Thurston}\\
    \textit{Radiology Registrar}
} 
\institute{
	\textit{Peninsula Radiology Academy}\\
    \textit{Plymouth}
}
\date[April 2018]{An introduction to automation with Python} % short date for footer



%% Content of slides %%

\begin{document}

    \begin{frame}[plain]
	    \titlepage
    \end{frame}


    \begin{frame}
	    \frametitle{Aims and objectives}
	    \framesubtitle{By the end of this session, you will be able to:}

	    \begin{itemize}
		    \item Part 1: Theory
			    \begin{itemize}
				    \item understand basic terms
				    \item have a good idea of some tasks that are possible
				    \item know where to learn more, if you are interested
			    \end{itemize}
	    \end{itemize}
	    \begin{itemize}
		    \item Part 2: Practice
			    \begin{itemize}
				    \item set up a Python environment
				    \item run some pre-written example programs
				    \item Write your own simple example programs
			    \end{itemize}
	    \end{itemize}
	    \begin{itemize}
		    \item questions welcome throughout
	    \end{itemize}
    \end{frame}
    %

    \begin{frame}
	    \frametitle{Why?}
	    \framesubtitle{how are these skills useful or relevant to me?}
	    \begin{itemize}
		    \item make your life easier: automate repetitive and boring tasks
		    \item get a new angle on a problem
			    \begin{itemize}
				    \item some simple problems are impossible without very basic programming skills
				    \item master your research and audit data
			    \end{itemize}
		    \item useful in understanding and implementing emerging radiology technologies
			    \begin{itemize}
				    \item AI, radiomics
				    \item knowledge enables you to navigate the hype -- useful at conferences and as a consultant
			    \end{itemize}
		    \item interesting and fun :) -- especially if you enjoy learning new things
	    \end{itemize}
    \end{frame}


    \begin{frame}
	    \frametitle{Definitions: programming language}
	    \framesubtitle{``formal language that specifies a set of instructions that can be used to produce various kinds of output''}
	    \begin{itemize}
		    \item like any language:
			    \begin{itemize}
				    \item spelling
				    \item grammatical rules
				    \item common useage
				    \item skill level -- basic to mastery 
					    \begin{itemize}
						    \item the basics can still be put to very good use
						    \item mastery may take a lifetime
					    \end{itemize}
			    \end{itemize}
	    \end{itemize}
			   \begin{itemize}
				   \item categorisation
					   \begin{itemize}
						   \item high level (Python, Java, etc) vs. lower level (C, assembly) vs. low level (machine code)
						   \item by programming paradigm: imperitive (procedural, object orientated), declarative (functional, logical), symbolic
					   \end{itemize} 
			   \end{itemize}
    \end{frame}

    \begin{frame}
	    \frametitle{Definitions: Python}
	    \framesubtitle{``The Python philosophy rejects exuberant syntax in favor of a simpler, less-cluttered grammar.''}
	    \begin{itemize}
		    \item based on the \textit{ABC language} --- optimised for ease of learning
			    \begin{itemize}
				    \item ``Simplicity is the ultimate sophistication'' \tiny{--- William Gaddis}
			    \end{itemize}
		    \item open source: 
			    \begin{itemize}
				    \item \textit{Python Software Foundation Licence (PSFL)} 
				    \item BSD style licence
			    \end{itemize}
		    \item massive community
			    \begin{itemize}
				    \item loads of extra functionality available as modules
				    \item domains include \textbf{data processing}, \textbf{artificial intelligence}, \textbf{image recognition}
				    \item help is easily available online for free if you get stuck
			    \end{itemize}
	    \end{itemize}
    \end{frame}

\defverbatim[colored]\lstA{
	\begin{lstlisting}[language=Python,basicstyle=\ttfamily,keywordstyle=\color{red}]
def drive_car(location):
	<precise instructions>

drive_car("50.376289, -4.143841")
drive_car("51.509865, -0.118092")
	\end{lstlisting}

}

    \begin{frame}
	    \frametitle{Definitions: abstraction}
	    \framesubtitle{``Ignore the characteristics that we don't need in order to concentrate on those that we do.''}
	    \begin{itemize}
		    \item in general:
			    \begin{itemize}
				    \item divide a programming problem into simpler, analogous pieces
				    \item solve the problem by combining solutions to simpler pieces
			    \end{itemize}
	    \end{itemize}
	    \begin{itemize}
		    \item Application Programming Interface (API)
			    \begin{itemize}
				    \item a set of clearly defined methods of communication between various software components
			    \end{itemize}
	    \end{itemize}
	    \lstA
    \end{frame}

\defverbatim[colored]\lstB{
	\begin{lstlisting}[language=Python,basicstyle=\ttfamily,keywordstyle=\color{red}]
print( *objects, sep=' ', end='\n',
	file=sys.stdout, flush=False)
	\end{lstlisting}

}

    \begin{frame}
	    \frametitle{Basic language syntax: hello world}
	    \framesubtitle{your first program}
	    {\scriptsize
	    ``The only way to learn a new programming language is by writing programs in it. The first program to write is the same for all languages:
	    Print the words: \textit{hello, world}
	    ...  With these mechanical details mastered, everything else is comparatively easy.''
	    \newline
	    -- Kernighan and Ritchie (``K\&R''), 1978
	    }

	    \begin{itemize}
		    \item the interpreter: python3
		    \item saving the file and running as a script: the \textit{``shebang''}
		    \item the print function: prints objects to a specified place (in our case, the screen \textit{stdout})
	    \end{itemize}
	    \lstB
    \end{frame}


\defverbatim[colored]\lstC{
	\begin{lstlisting}[language=Python,basicstyle=\ttfamily,keywordstyle=\color{red}]
def test_function():
	'''Runs a test.'''
	print("Testing... 1, 2, 3")

# run the test
test_function()
	\end{lstlisting}

}

    \begin{frame}
	    \frametitle{Basic language syntax: comments and docstrings}
	    \framesubtitle{ensure maximum readibility... for others (and yourself)}
	    \begin{itemize}
		    \item a \# character makes the interpreter ignore the rest of the line
		    \item at the start of a function definition, a \textit{docstring} describes the function
	    \end{itemize}
	    \lstC
    \end{frame}






\defverbatim[colored]\lstD{
	\begin{lstlisting}[language=Python,basicstyle=\ttfamily,keywordstyle=\color{red}]
my_var = "value"
denominator = 7 
GLOBAL_VARIABLES = "all caps"
	\end{lstlisting}

}


    \begin{frame}
	    \frametitle{Basic language syntax: variables \& data}
	    \framesubtitle{data are the fundamental building blocks of any program}
	    \begin{itemize}
		    \item variables allow storage of data by a name
		    \item be careful how you name them!
			    \begin{itemize}
				    \item clear and consise names: easy to read, descriptive
				    \item don't clash with other names in use (including language reserved keywords
				    \item Python style guide recommendation: variable names should be lowercase, with words separated by underscores as necessary to improve readability.
					    Uppercase global variable names.
			    \end{itemize}
	    \end{itemize}
	    \lstD
    \end{frame}

\defverbatim[colored]\lstE{
	\begin{lstlisting}[language=Python,basicstyle=\ttfamily,keywordstyle=\color{red}]
body_parts = ["hand", "elbow", "shoulder"]
months = ("Jan", "Feb", "Mar", ...) 
empty_list = list()

>>> print(months[0]) # 0-indexed
Jan
	\end{lstlisting}

}


    \begin{frame}
	    \frametitle{Basic language syntax: lists, iterables}
	    \framesubtitle{a list is a type of sequence}
	    \begin{itemize}
		    \item an iterable is an object with multiple elements:
			    \begin{itemize}
				    \item including \textit{lists}, \textit{tuples}, \textit{sets}
			    \end{itemize}
		    \item accessing multiple times can allow you to perform an operation on every element
	    \end{itemize}
	    \lstE
    \end{frame}

\defverbatim[colored]\lstF{
	\begin{lstlisting}[language=Python,basicstyle=\ttfamily,keywordstyle=\color{red}]
>>> 1 == 1
True
>>> 10 <= 0
False
\end{lstlisting}

}

    \begin{frame}
	    \frametitle{Basic language syntax: Boolean statements}
	    \framesubtitle{True, False, 1, 0}
	    \begin{itemize}
		    \item Boolean algebra named after George Boole, 1815 -- 1864
		    \item the lowest construct that a computer understands
		    \item the return value from a test statement is either True or False
	    \end{itemize}
	    \lstF

    \end{frame}


\defverbatim[colored]\lstI{
\begin{lstlisting}[language=Python,basicstyle=\ttfamily,keywordstyle=\color{red}]
if <statement>:
	<indented code block>
elif <another statement>:
	<indented code>
else:
	<more code>

<statement> == True # activates 
<statement> == False # skips
\end{lstlisting}

}


    \begin{frame}
	    \frametitle{Basic language syntax: conditional statements}
	    \framesubtitle{perform a different action depending on the result of a test}
	    \begin{itemize}
		    \item control which part of the program is activated, depending on a test 
	    \end{itemize}
	    \lstI

    \end{frame}

\defverbatim[colored]\lstJ{
\begin{lstlisting}[language=Python,basicstyle=\ttfamily,keywordstyle=\color{red}]
for item in <iterable>:
	<code block>

while <condition>:
	<code block>
\end{lstlisting}

}


    \begin{frame}
	    \frametitle{Basic language syntax: loops}
	    \framesubtitle{repeat a process over and over until a condition is met}
	    \begin{itemize}
		    \item \textit{iteration}: perform an operation on multiple elements
			    \begin{itemize}
				    \item each file in a directory
				    \item each DICOM image in a list
			    \end{itemize}
		    \item in Python --- \textit{for loop} and \textit{while loop}:
	    \end{itemize}
	    \lstJ
    \end{frame}


\defverbatim[colored]\lstK{
\begin{lstlisting}[language=Python,basicstyle=\ttfamily,keywordstyle=\color{red}]
# get some standard input
user_input = input()

# prints all the items in the argument vector
import sys
for item in sys.argv:
   print(item)
\end{lstlisting}
}


    \begin{frame}
	    \frametitle{Basic language syntax: user input}
	    \framesubtitle{store data to operate on}
	    How do I get data in to my program?
	    \begin{itemize}
		    \item \textit{stdin}: ``standard input''
		    \item \textit{args}: command line arguments
	    \end{itemize}
	    \lstK
    \end{frame}

   \defverbatim[colored]\lstL{
	   \begin{lstlisting}[language=Python,basicstyle=\ttfamily,keywordstyle=\color{red}]
# use the os module to find all the files
import os
files = os.walk('.')
\end{lstlisting}

   }

   \defverbatim[colored]\lstM{
\begin{lstlisting}[language=Bash,basicstyle=\ttfamily,keywordstyle=\color{red}]
# install a new python module
$ pip3 install tensorflow
\end{lstlisting}

}

    \begin{frame}
	    \frametitle{Basic language syntax: using external modules}
	    \framesubtitle{add features to your program}
	    \begin{itemize}
		    \item allows you to take advantage of built in and community contributed external code
		    \item pip: tool to collect and install external modules from internet repository --- minimal effort required
	    \end{itemize}
	    \lstL
	    \lstM
    \end{frame}

    \begin{frame}
	    \frametitle{Designing your program: drawing board}
	    \framesubtitle{think about the process}
	    {
		    \scriptsize
		    ``Computer science is no more about computers than astronomy is about telescopes.''
		    \newline
		    -- Dijkstra
		    }
	    \begin{itemize}
		    \item paper and pen
		    \item get the steps right before you start and the rest will be easy
		    \item the programming language syntax is of secondary importance
	    \end{itemize}
    \end{frame}

    \begin{frame}
	    \frametitle{Possibilities: file renaming}
	    \framesubtitle{Automate a boring task}
	    For example, your digital camera has produced a directory of files named \textbf{IMG\_3624.JPG} etc...
	    You would prefer them named Greece\-2018\_xxx.jpg
	    \newline
	    Think about the steps that would be required for this:
	    \begin{enumerate}
		    \item select the appropriate directory location
			    \begin{itemize}
				    \item either hard-coded or user-selectable
				    \item can you trust the user to enter a valid location?
			    \end{itemize}
		    \item build a list of all the items you want renamed
		    \item go through each item, performing the rename operation
			    \begin{itemize}
				    \item format for renaming? a counter might be needed
			    \end{itemize}
		    \item exit
	    \end{enumerate}
    \end{frame}

    \begin{frame}
	    \frametitle{Possibilities: remove all duplicate patients from a spreadsheet}
	    \framesubtitle{Automate a boring task}
	    \begin{enumerate}
		    \item Precisely define the task: how do you define a duplicate row?
		    \item import the data to Python using the help of an external module
			    \begin{itemize}
				    \item method depends on the format of the data -- \textit{Excel}, \textit{comma separated value file}
			    \end{itemize}
		    \item loop through all the records removing those that fit (or don't fit) the criteria
		    \item write to a new file
		    \item exit
	    \end{enumerate}
    \end{frame}

    \begin{frame}
	    \frametitle{Possibilities: find all phone numbers in a text document}
	    \framesubtitle{Regular expressions}
	    \begin{enumerate}
		    \item What does a phone number look like?  
			    \begin{itemize}
				    \item country code [optional], area code, number
				    \item may include dashes or spaces
			    \end{itemize}
		    \item Design a pattern (\textit{regular expression}) that fits the above
		    \item Test it, looking for false positives and false negatives
	    \end{enumerate}
    \end{frame}


    \begin{frame}
	    \frametitle{Modules: useful extensions}
	    \framesubtitle{A few selected modules out of hundreds}
	    \begin{itemize}
		    \item \textit{Tensorflow}: open source machine learning library from Google
		    \item \textit{Pandas, numpy}: data analysis toolkits
		    \item \textit{Requests}: allows you to get webpages and text from the web into your program
		    \item \textit{Scrapy}: scrape data from the web
		    \item \textit{wxpython}: build a graphical program
	    \end{itemize}
    \end{frame}



    \begin{frame}
	    \frametitle{Resources: books}
	    \framesubtitle{Learning resources available on the internet}
	    \begin{itemize}
		    \item Automate the boring stuff with Python: \url{https://automatetheboringstuff.com/}
		    \item Another free online tutorial: \url{https://python-textbok.readthedocs.io/en/1.0/}
	    \end{itemize}
    \end{frame}


    \begin{frame}
	    \frametitle{Resources: interactive tutorials}
	    \framesubtitle{Learning resources available on the internet}
	    \begin{itemize}
		    \item Massive open online courses: some free, some cost
			    \begin{itemize}
				    \item EdX: \url{https://www.edx.org/course?search_query=python}
				    \item Coursera: \url{https://www.coursera.org/courses?languages=en&query=python&userQuery=python}
				    \item Udacity: \url{https://eu.udacity.com/}
				    \item udemy \url{https://www.udemy.com/}
			    \end{itemize}
	    \end{itemize}
    \end{frame}

    \begin{frame}
	    \frametitle{Resources: practice to improve your skills}
	    \framesubtitle{Thousands of resources are available}
	    \begin{itemize}
		    \item coding websites
			    \begin{itemize}
				    \item \url{https://www.hackerrank.com/domains/python}
				    \item \url{https://www.codecademy.com/tracks/python}
			    \end{itemize}
		    \item open access data
			    \begin{itemize}
				    \item \url{https://data.gov.uk/}
			    \end{itemize}
		    \item read other code on \url{https://github.com}
		    \item contribute to an open source project
			    \begin{itemize}
				    \item Horos
				    \item Orthanc
				    \item OpenCV
				    \item Python
			    \end{itemize}
	    \end{itemize}
    \end{frame}

    \begin{frame}
	    \frametitle{Resources: getting help}
	    \framesubtitle{Help from the community}
	    \begin{itemize}
		    \item Stackoverflow: \url{https://stackoverflow.com/tags/python/info}
		    \item Internet relay chat: \#python on irc.freenode.net
		    \item Mailing lists: \url{https://www.python.org/community/lists/}
	    \end{itemize}
    \end{frame}
   \begin{frame}
	   \frametitle{Summary}
	    \framesubtitle{Topics covered}
	    \begin{itemize}
		    \item basic definitions and background
		    \item language syntax
			    \begin{itemize}
				    \item variables
				    \item control structures
				    \item user input
			    \end{itemize}
		    \item program design
		    \item some uses
		    \item help and resources
	    \end{itemize}
   \end{frame}




\end{document}
