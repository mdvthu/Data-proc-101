%% If you have any problems using this template, please contact the author: %%
%% timhosgood@gmail.com %%

\documentclass{beamer}
\usepackage[utf8]{inputenc}
\usepackage{charter}
\usepackage{tikz}
\usepackage{graphicx}
\usepackage{amsmath}
\usepackage{amssymb}
\usepackage{epigraph}
\usepackage{trace}

\usepackage{listings}

%% Title slide formatting %%

\pgfdeclareimage[width=\paperwidth]{titlebackground}{images/title-slide-background.png}
\setbeamerfont{subtitle}{size=\tiny}
\setbeamertemplate{title page}{
	\begin{picture}(0,0)
		\put(-28.5,-163){%
			\pgfuseimage{titlebackground}
		}
		\put(0,-75){%
			\begin{minipage}[b][4.5cm][t]{0.5\textwidth}
				\color{black}
				\usebeamerfont{title}
				{\inserttitle\\[0.9cm]}
				\usebeamerfont{subtitle}
				{\insertauthor\par}
				{\insertinstitute\\[0.3cm]}
				{\insertdate}
			\end{minipage}
		}
	\end{picture}
}



%% General slide formatting %%

\definecolor{nhsdarkblue}{RGB}{0,48,135}

\pgfdeclareimage[width=1.4cm]{phntlogo}{images/phnt-logo.png}

\setbeamertemplate{headline}
{%
	\begin{picture}(0,0)
		\put(314,-50){%
			\pgfuseimage{phntlogo}
		}
		\put(20,-55){%
			\rule{320pt}{0.4pt}
		}
	\end{picture}
}

\setbeamertemplate{frametitle}
{%
	\begin{picture}(0,0)
		\put(-8,-10){%
			\normalsize\color{nhsdarkblue}\insertframetitle
		}
		\put(-7,-20){%
			\tiny\color{nhsdarkblue}\insertframesubtitle
		}
	\end{picture}
}

\setbeamertemplate{footline}
{%
	\begin{picture}(0,0)
		\put(20,30){%
			\rule{320pt}{0.4pt}
		}
		\put(100,14){%
			\color{nhsdarkblue}\insertshortdate
		}
		\put(160,14){%
			\color{nhsdarkblue}\insertshorttitle
		}
		\put(337,14){%
			\color{nhsdarkblue}\insertpagenumber
		}
	\end{picture}%
}



%% Information (author, title, etc.) %%

\title[Programming workshop]{Programming workshop: a brief introduction}

\author{
	\sc{Mark Thurston}\\
    \textit{Radiology Registrar}
} 
\institute{
	\textit{Peninsula Radiology Academy}\\
    \textit{Plymouth}
}
\date[April 2018]{An introduction to automation with Python} % short date for footer



%% Content of slides %%

\begin{document}

    \begin{frame}[plain]
	    \titlepage
    \end{frame}


    \begin{frame}
	    \frametitle{Aims and objectives}
	    \framesubtitle{By the end of this session, you will be able to:}

	    \begin{itemize}
		    \item Part 1: Theory
			    \begin{itemize}
				    \item understand basic terms
				    \item have a good idea of some tasks that are possible
				    \item know where to learn more, if you are interested
			    \end{itemize}
	    \end{itemize}
	    \begin{itemize}
		    \item Part 2: Practice
			    \begin{itemize}
				    \item set up a Python environment
				    \item run some pre-written example programs
				    \item Write your own simple example programs
			    \end{itemize}
	    \end{itemize}
	    \begin{itemize}
		    \item ask questions at any point if unclear
	    \end{itemize}
    \end{frame}
    %

    \begin{frame}
	    \frametitle{Why?}
	    \framesubtitle{how are these skills useful or relevant to me?}
	    \begin{itemize}
		    \item make your life easier: automate repetitive and boring tasks
		    \item get a new angle on a problem
			    \begin{itemize}
				    \item some simple problems are impossible without very basic programming skills
				    \item master your research data
			    \end{itemize}
		    \item useful in understanding and implementing emerginig radiological technologies
			    \begin{itemize}
				    \item AI, radiomics
				    \item knowledge enables you to navigate the hype -- useful at conferences and as a consultant
			    \end{itemize}
		    \item interesting and fun :) -- a huge field if you enjoy learning new things
	    \end{itemize}
    \end{frame}


    \begin{frame}
	    \frametitle{Definitions: programming language}
	    \framesubtitle{``formal language that specifies a set of instructions that can be used to produce various kinds of output''}
	    \begin{itemize}
		    \item like any language:
			    \begin{itemize}
				    \item spelling
				    \item grammatical rules
				    \item common useage
				    \item skill level -- basic to mastery -- the basics can still be put to very good use, mastery may take a lifetime
			    \end{itemize}
		    \item categorisation
			    \begin{itemize}
				    \item high level (Python, Java, etc) vs. lower level (C, assembly) vs. low level (machine code)
				    \item by programming paradigm: imperitive (procedural, object orientated), declarative (functional, logical), symbolic
			    \end{itemize} 
	    \end{itemize}
    \end{frame}

    \begin{frame}
	    \frametitle{Definitions: Python}
	    \framesubtitle{``The Python philosophy rejects exuberant syntax in favor of a simpler, less-cluttered grammar.''}
	    \begin{itemize}
		    \item easy to learn: based on the ``ABC'' language - a research project 
			    \begin{itemize}
				    \item ``Simplicity is the ultimate sophistication'' --- William Gaddis
			    \end{itemize}
		    \item open source
		    \item massive community
			    \begin{itemize}
				    \item loads of extra functionality available as modules
				    \item data processing, artificial intelligence, image recognition
				    \item help is easily available online for free if you get stuck
			    \end{itemize}
	    \end{itemize}
    \end{frame}

    \begin{frame}
	    \frametitle{Definitions: abstraction}
	    \framesubtitle{``Ignore the characteristics that we don't need in order to concentrate on those that we do.''}
	    \begin{itemize}
		    \item in general:
			    \begin{itemize}
				    \item Divide a programming problem into simpler, analogous pieces
				    \item Solve the problem by combining solutions to simpler pieces
			    \end{itemize}
		    \item Application Programming Interface (API)
			    \begin{itemize}
				    \item a set of clearly defined methods of communication between various software components
			    \end{itemize}
	    \end{itemize}
    \end{frame}


    \begin{frame}
	    \frametitle{Basic language syntax: hello world}
	    \framesubtitle{your first program}

	    ``The only way to learn a new programming language is by writing programs in it. The first program to write is the same for all languages:
	    Print the words: \textit{hello, world}
	    ...  With these mechanical details mastered, everything else is comparatively easy.''
	    Kernighan and Ritchie (``K\&R''), 1978

	    \begin{itemize}
		    \item the interpreter: python3
		    \item the development environment: idle
		    \item saving the file and running as a script: the \textit{``shebang''}
	    \end{itemize}
    \end{frame}

    \begin{frame}
	    \frametitle{Basic language syntax: variables \& data}
	    \framesubtitle{data are the fundamental building blocks of any program}
	    \begin{itemize}
		    \item Allows storage of data by a name:
			    \begin{itemize}
				    \item my\_var = "value"
				    \item denominator = 7
			    \end{itemize}
		    \item Be careful how you name them!
			    \begin{itemize}
				    \item clear and consise names: easy to read, descriptive
				    \item don't clash with other names in use (including language reserved keywords
				    \item Python style guide recommendation: variable names should be lowercase, with words separated by underscores as necessary to improve readability
				    \item GLOBAL\_VARIABLES = "all caps"
			    \end{itemize}
	    \end{itemize}
    \end{frame}

    \begin{frame}
	    \frametitle{Basic language syntax: lists, iterables}
	    \framesubtitle{a list is a type of sequence}
	    \begin{itemize}
		    \item an iterable is an object with multiple elements
		    \item accessing multiple times can allow you to perform an operation on every element
	    \end{itemize}
    \end{frame}

\defverbatim[colored]\lstI{
\begin{lstlisting}[language=Python,basicstyle=\ttfamily,keywordstyle=\color{red}]
if <statement>:
	<indented code block>
elif <another statement>:
	<indented code>
else:
	<more code>
\end{lstlisting}

}


    \begin{frame}
	    \frametitle{Basic language syntax: conditional statements}
	    \framesubtitle{perform a different action depending on the result of a test}
	    \begin{itemize}
		    \item in Python:
	    \end{itemize}
	    \lstI

    \end{frame}

\defverbatim[colored]\lstJ{
\begin{lstlisting}[language=Python,basicstyle=\ttfamily,keywordstyle=\color{red}]
for item in <iterable>:
	<code block>

while <condition>:
	<code block>
\end{lstlisting}

}


    \begin{frame}
	    \frametitle{Basic language syntax: loops}
	    \framesubtitle{repeat a process over and over until a condition is met}
	    \begin{itemize}
		    \item \textit{iteration}
		    \item in Python --- \textit{for loop} and \textit{while loop}:
	    \end{itemize}
	    \lstJ
    \end{frame}

    \begin{frame}
	    \frametitle{Basic language syntax: user input}
	    \framesubtitle{store data to operate on}
	    \begin{itemize}
		    \item \textit{stdin}
		    \item \textit{args}: command line arguments
	    \end{itemize}
    \end{frame}


    \begin{frame}
	    \frametitle{Basic language syntax: using external modules}
	    \framesubtitle{add features to your program}
	    \begin{itemize}
		    \item allows you to take advantage of community contributed code
		    \item pip: tool to collect and install external modules from internet repository
	    \end{itemize}
    \end{frame}

    \begin{frame}
	    \frametitle{Designing your program: drawing board}
	    \framesubtitle{think about the process}
		    ``Computer science is no more about computers than astronomy is about telescopes.'' --- Dijkstra
	    \begin{itemize}
		    \item paper and pen
		    \item get the steps right before you start and the rest will be easy
	    \end{itemize}
    \end{frame}

    \begin{frame}
	    \frametitle{Possibilities: file renaming}
	    \framesubtitle{Automate a boring task}
	    \begin{itemize}
		    \item 
	    \end{itemize}
    \end{frame}

    \begin{frame}
	    \frametitle{Possibilities: remove all duplicate patients from a spreadsheet}
	    \framesubtitle{Automate a boring task}
	    \begin{itemize}
		    \item 
	    \end{itemize}
    \end{frame}

    \begin{frame}
	    \frametitle{Possibilities: find all phone numbers in a text document}
	    \framesubtitle{Regular expressions}
	    \begin{itemize}
		    \item 
	    \end{itemize}
    \end{frame}


    \begin{frame}
	    \frametitle{Possibilities: recognise the type of flower in an image}
	    \framesubtitle{Tensorflow}
	    \begin{itemize}
		    \item 
	    \end{itemize}
    \end{frame}

    \begin{frame}
	    \frametitle{Modules: useful extensions}
	    \framesubtitle{A few selected modules out of hundreds}
	    \begin{itemize}
		    \item Tensorflow
		    \item NLTK
		    \item Pandas and numpy
		    \item Requests
		    \item pygtk
		    \item scrapy
	    \end{itemize}
    \end{frame}



    \begin{frame}
	    \frametitle{Resources: free tutorials}
	    \framesubtitle{Learning resources available on the internet}
	    \begin{itemize}
		    \item 
		    \item \textit{massive open online courses}
			    \begin{itemize}
				    \item EdX, Coursera
			    \end{itemize}
	    \end{itemize}
    \end{frame}

    \begin{frame}
	    \frametitle{Resources: practice to improve your skills}
	    \framesubtitle{Thousands of resources are available}
	    \begin{itemize}
		    \item coding websites
			    \begin{itemize}
				    \item Hacker Rank
			    \end{itemize}
		    \item open access data
		    \item contributing to an open source project
			    \begin{itemize}
				    \item Horos, Orthanc, OpenCV, Python
			    \end{itemize}
	    \end{itemize}
    \end{frame}

    \begin{frame}
	    \frametitle{Resources: getting help}
	    \framesubtitle{Help from the community}
	    \begin{itemize}
		    \item Stackoverflow
		    \item Internet relay chat: \#python on irc.freenode.net
		    \item Mailing lists:
	    \end{itemize}
    \end{frame}


\end{document}
